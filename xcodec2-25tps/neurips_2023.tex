\documentclass[preprint]{article}

% if you need to pass options to natbib, use, e.g.:
%     \PassOptionsToPackage{numbers, compress}{natbib}
% before loading neurips_2023


% ready for submission
\usepackage{neurips_2023}


% to compile a preprint version, e.g., for submission to arXiv, add add the
% [preprint] option:
%     \usepackage[preprint]{neurips_2023}


% to compile a camera-ready version, add the [final] option, e.g.:
%     \usepackage[final]{neurips_2023}


% to avoid loading the natbib package, add option nonatbib:
%    \usepackage[nonatbib]{neurips_2023}

\usepackage{bookmark}
\usepackage[utf8]{inputenc} % allow utf-8 input
\usepackage[T1]{fontenc}    % use 8-bit T1 fonts
\usepackage{hyperref}       % hyperlinks
\usepackage{url}            % simple URL typesetting
\usepackage{booktabs}       % professional-quality tables
\usepackage{amsfonts}       % blackboard math symbols
\usepackage{nicefrac}       % compact symbols for 1/2
\usepackage{microtype}      % microtypography
\usepackage{xcolor}         % colors
\usepackage{listings}
\usepackage{cite}
\usepackage{graphicx}
\usepackage{hyperref}
\usepackage{amsmath}
\usepackage{algpseudocode}
\usepackage{amsmath}
\usepackage{algorithm}
\usepackage{caption}
\usepackage{pdflscape}
\usepackage{tabularx}

\lstset{
  breaklines=true,
  basicstyle=\ttfamily\small,
  commentstyle=\color{green!40!black},
  keywordstyle=\color{blue},
  numberstyle=\tiny\color{gray},
  numbers=none,
  frame=single,
  breaklines=true,
  breakatwhitespace=true,
  captionpos=b,
  showstringspaces=false,
  columns=fullflexible,
}

\hypersetup{
    pdftitle={Improving X-Codec-2.0 for Multi-Lingual Speech: 25 Hz Latent Rate and 24 kHz Sampling},
    pdfauthor={Husein Zolkepli},
    pdfsubject={Audio and Speech Processing},
    pdfkeywords={speech models, deep learning},
    colorlinks=true,
    linkcolor=blue,
    citecolor=blue,
    urlcolor=blue
}
\title{Improving X-Codec-2.0 for Multi-Lingual Speech: 25 Hz Latent Rate and 24 kHz Sampling}
\author{
  Husein Zolkepli\thanks{husein.zol05@gmail.com}
}

\begin{document}

\maketitle

\begin{abstract}

X-Codec-2.0 has shown strong performance in neural audio compression and multilingual speech modeling, operating at a 50 Hz latent rate and a 16 kHz sampling rate using frozen HuBERT features. While effective, this configuration limits efficiency and audio fidelity.
In this work, we explore a simple yet effective modification: introducing additional pooling and increasing the decoder hop size. This reduces the latent rate from 50 Hz to 25 Hz and simultaneously raises the output sampling rate from 16 kHz to 24 kHz, improving temporal efficiency and perceptual quality without altering the core architecture.
Evaluated on the multilingual Common Voice 17 test set, the proposed configuration achieves a 0.5 MOS improvement over the original X-Codec-2.0 baseline. The source code, checkpoints and generation comparison released at \href{https://xcodec2-25tps.github.io}{xcodec2-25tps.github.io}.

\end{abstract}

\section{Introduction}

.

\section{Method}

.

\section{Experiments}

.

\section{Results}

.

\section{Acknowledgement}

Special thanks to my wife for her patience and for not getting too upset about the electricity bill
caused by running two RTX 3090 Ti GPUs around the clock for two months.

\bibliography{neurips_2023}{}
\bibliographystyle{unsrt}

\end{document}